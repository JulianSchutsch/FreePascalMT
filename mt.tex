\documentclass{beamer}
\usepackage[utf8]{inputenc}
\usepackage{color}
\usepackage{verbatim}
\title{Multi threading with FreePascal}
\author{Julian Schutsch}
\date{13. July 2013}
\setbeamertemplate{frametitle}
{
   \begin{center}
    \insertframetitle
   \end{center}
}
\useoutertheme{infolines}
\begin{document}
\begin{frame}
\frametitle{Multi threading with FreePascal}
\begin{itemize}
 \item Introduction
 \item What FreePascal offers
 \item Conditional variables
 \item Queues
 \item Example Patterns
 \item GUI programs
 \item Warnings
 \item Thanks
\end{itemize}
\end{frame}

\begin{frame}
\frametitle{Introduction}
\begin{center}
Reasons for multi threading
\end{center}
\begin{itemize}
 \item Solving computation heavy problems
 \item Scalable handling of events (e.g. network)
 \item Keeping applications responsive
\end{itemize}
\end{frame}

\begin{frame}
\frametitle{Introduction}
\begin{center}
FreePascal?
\end{center}
\begin{itemize}
 \item FreePascal (as many others) not made for parallel computation
 \item RTLs offer some multi threading support
\end{itemize}
\vspace{1cm}
\begin{centering}
All multi threading will be very low level.\par
\end{centering}
\end{frame}

\begin{frame}
\frametitle{What FreePascal offers}
\begin{tabular}{l|l|l}
 \color{green}Threads & \color{green}RTL (system,classes) & \color{green}2 variants\\
 \color{green}ThreadPools & \color{green}Packages (MTProcs) & \color{green}Ok\\
 \color{green}Atomic operations & \color{green}RTL (system) & \color{green}Ok\\
 \color{green}Memory barriers & \color{green}RTL (system) & \color{green}Ok\\
 \color{green}Events & \color{green}RTL (system) & \color{green}Ok\\
 \color{orange}Mutex & \color{orange}RTL (system) & \color{orange}Conditional variables\\
 \color{orange}Spinlock & \color{orange}- & \color{orange}$\to$Atomic operations\\
 \color{red}Conditional variables & \color{red}- & \color{red}\\
 \color{red}Semaphores & \color{red}- & \color{red}\\
 \color{red}Read\&Write locks & \color{red}- & \color{red}\\
 \color{red}Volatile & \color{red}- &\color{red}Not good\\
 \color{red}Barrier & \color{red}- &\color{red}\\
\end{tabular}
\end{frame}

\begin{frame}
\frametitle{Conditional variables}
\begin{center}
Why?
\end{center}
\begin{itemize}
 \item Combination of events and mutex
 \item Signaling one or all waiting threads
 \item Allows easy implementation and proof of queues and events
 \item But: Windows only has them starting with Vista
\end{itemize}
\begin{center}
Combination with mutex
\end{center}
\begin{itemize}
 \item Critical section (abstraction) not accessible $\Rightarrow$ Reimplement
\end{itemize}
\end{frame}

\begin{frame}
\frametitle{Queues}
\begin{center}
 What?
\end{center}
\begin{itemize}
 \item Thread safe, first in$\to$ first out abstract data type
\end{itemize}
\begin{center}
 Why?
\end{center}
\begin{itemize}
 \item Good abstraction for separation of threads responsibilities
 \item Can be combined with shared memory (ownership)
 \item Can proof absence of dead locks, if certain rules are followed
 \item Number of design patterns available
 \item Several variants for different purposes
\end{itemize}
\end{frame}

\begin{frame}
\frametitle{Unbounded Queue}
\begin{itemize}
 \item Queue protected by a lock
 \item Sending is non blocking
 \item Receiving is blocking
 \item Single conditional variable wakes up receiving thread
\end{itemize}
\end{frame}
\begin{frame}
\frametitle{Bounded Queue}
\begin{itemize}
 \item Circular buffer protected by a lock
 \item Sending is blocking, if buffer is full
 \item Receiving is blocking
 \item "Written" Conditional variable wakes up receiving thread
 \item "Read" Conditional variable wakes up one blocked sending thread
 \item No memory allocation/release for send/receive
 \item Possible variation: Broadcast on empty instead of signal on read
\end{itemize}
\end{frame}

\begin{frame}
\frametitle{Example patterns: Producer-Consumer}
\begin{enumerate}
 \item Multi sender, Multi receiver $\Rightarrow$ Queue
 \item Single sender, Single receiver $\Rightarrow$ Non blocking approaches (volatile?)
 \item Interrupts $\Rightarrow$ Non blocking, protected read
\end{enumerate}
\end{frame}

\begin{frame}
\frametitle{Example patterns: Leader-Follower}
\begin{itemize}
 \item Multi threaded processing of events without queuing
\end{itemize}
\begin{enumerate}
 \item Create pool of threads
 \item Select leader, rest goes to wait
 \item Leader waits for event
 \item On event $\Rightarrow$ give up leadership (send signal)
 \item Once processing is completed, rejoin as follower
\end{enumerate}
\end{frame}

\begin{frame}
\frametitle{Example patterns: Active-Object}
\begin{itemize}
 \item Requires futures
 \item Asynchronous execution of methods in a designated thread
\end{itemize}
\end{frame}

\begin{frame}
\frametitle{GUI}
\begin{itemize}
 \item Main loop, waiting for events (not portable)
 \item GUI should never wait
 \item Asynchronous calls the best alternative (also not portable)
 \item Timers are a possible "hack" to implement them
 \item FreePascal offers synchronize, if main loop supports it
 \item Many design patterns must be adjusted
\end{itemize}
\end{frame}

\begin{frame}
\frametitle{Asynchronous Future}
% http://docs.scala-lang.org/sips/pending/futures-promises.html
\end{frame}

\begin{frame}
\frametitle{Model-View-Controller}
\begin{itemize}
 \item View must access/iterate through model
 \item What if changes to the model happen in another thread?
 \item If model is not trivial, either synchronize or copy
\end{itemize}
\end{frame}

\begin{frame}[fragile]
\frametitle{Warnings: Compiler reordering optimizations}

\begin{center}
Example 1
\end{center}
\begin{tabular}{p{5.7cm} p{5.7cm}}
What you write
&
What the compiler understands\\
\begin{verbatim}
while x<>0 do;
\end{verbatim}
&
\begin{verbatim}
if x<>0 then while True do;
\end{verbatim}
\end{tabular}
\begin{center}
This kind of code could be used in a two phase shutdown.\\
Could be improved by conditional variables.\\
Or you are lucky (enough code in the loop?).\\
\end{center}
\end{frame}

\begin{frame}[fragile]
\frametitle{Warnings: Compiler reordering optimizations}

\begin{center}
Example 2
\end{center}
\begin{tabular}{p{5.7cm} p{5.7cm}}
What you write
&
What the compiler understands\\
\begin{verbatim}
x:=0;
while true do
begin
  if x=1 then writeln("Ok");
end;
\end{verbatim}
&
\begin{verbatim}
x:=0;
while true do;
\end{verbatim}
\end{tabular}

\begin{center}
I admit, thats at the moment more gcc with -O3.\\
But are you willing to risk it? Volatile is really missing!
\end{center}
\end{frame}

\begin{frame}[fragile]
\frametitle{Warnings: Out-of-order execution}
\begin{center}
Example
\end{center}
\begin{tabular}{p{5.7cm} p{5.7cm}}
What you write
&
What might happen\\
\begin{verbatim}
x:=1;
y:=2;
\end{verbatim}
&
\begin{verbatim}
y:=2;
x:=1;
\end{verbatim}
\end{tabular}
\begin{center}
That is a sure way to sabotage many lock free algorithms.\\
Use FPCs memory barriers (in this case \tt{writebarrier})
\end{center}
\end{frame}

\begin{frame}
\frametitle{Thanks}
\end{frame}

\end{document}

% Scatter -> Map -> Reduce
% Blocked Networking
% > Device driver waiting
% Separation of GUI and the rest

% Queues : Blocking against Non-Blocking
%          Bounded vs Non Bounded

% Shared Memory
%  - Ownership transfer

% Memory barrier issues
% Memory bandwidth issues

% Distributed design paradigms
% => Concurrency pattern
%    http://en.wikipedia.org/wiki/Concurrency_pattern

% Windows Native vs Posix
%  Semaphores
%  Mutex
%  Spinlock
%  Read/Write Locks

% Blocking, but not dead,
%  Periodic wakeup, Signaling